\section{Introduction}
Gravitational compression in the core of stars fuse lighter elements into heavier ones, which persist until the energy generated from the nuclear reactions is able to withhold it's own gravity. Simpler nuclear processes, such as hydrogen and helium burning, are well understood with straightforward reactions. However, as a star evolves into later stages—like carbon, neon, oxygen, and silicon burning—nuclear processes become increasingly complex. Detailed networks are essential to model these stages, capturing the diverse reactions that govern elemental synthesis and distribution. During the burning stages, the stellar plasma is continually mixed between its convective and radiative zones, a process known as Convective Boundary Mixing (CBM). This influences nuclear reaction rates, making it vital for understanding stellar evolution. Although nuclear networks and CBM define what occurs in the core of a star, their initial conditions have a huge impact on them like mass and metallicity. For instance, massive stars, with higher core temperatures, undergo rapid and intense nuclear burning, leading to shorter lifetimes and explosive end stages like supernovae. In contrast, lower-mass stars burn fuel more slowly, resulting in longer lifetimes and gentler outcomes, such as white dwarfs. Hence models with various masses and metallicities are included in the study.
