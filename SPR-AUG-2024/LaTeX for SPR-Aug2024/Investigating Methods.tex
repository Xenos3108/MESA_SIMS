\subsection{Methods for Investigating Stellar Evolution}

\subsubsection{Kippenhahn Diagram}
The Kippenhahn Diagram is an invaluable tool for probing the internal structure of stars and understanding their evolutionary processes. In this diagram, the stellar age is plotted against the radial mass coordinate, effectively mapping the star’s internal layers over time. The diagram uses a color gradient to distinguish between convective and radiative zones, thus illustrating how energy is generated and transported within the star. Early in a star’s life, for example, a prominent convective core is often visible, indicating regions where vigorous mixing and energy transport occur. As the star evolves, shifts in these zones reflect changes in nuclear burning and energy transport mechanisms.

\begin{figure}[h!]
   \centering
   \includegraphics[width=0.75\textwidth]{Varying Nuclear Networks/EXkip.png}
   \caption{Kippenhahn Diagram illustrating the evolution of the star's internal structure, with convective and radiative zones indicated by a color gradient over time.}
   \label{fig:EXKip}
\end{figure}

\vspace{-2em}
\subsubsection{Hertzsprung-Russell Diagram (HRD)}
The Hertzsprung-Russell Diagram (HRD) is a fundamental diagnostic tool in astrophysics that plots a star’s luminosity against its surface temperature (or spectral class). This diagram provides a snapshot of a star’s evolutionary trajectory over time. Stars begin their lives on the Zero Age Main Sequence (ZAMS), where they steadily fuse hydrogen in their cores. As a star evolves, structural changes—such as the depletion of core hydrogen—cause its position on the HRD to shift dramatically. For instance, a steep increase in luminosity is typically observed as the star exhausts its hydrogen fuel; the contracting core heats up rapidly, resulting in a sharp rightward turn on the HRD. This characteristic change marks the transition from main-sequence evolution to later stages.

\begin{figure}[h!]
   \centering
   \includegraphics[width=0.9\textwidth]{Varying Nuclear Networks/EXHRD.png}
   \caption{Evolution of the models in the HRD, showing the transition from the Zero Age Main Sequence (ZAMS) through core hydrogen exhaustion and subsequent evolution.}
   \label{fig:EXHRD}
\end{figure}
